%!TEX root = PhD_Thesis.tex
\chapter{Introduction}

Humans perceive information using the sense of touch through integration of kinesthetic and cutaneous tactile sensory information. Haptic devices aim to render a realistic sense of touch by replicating these kinesthetic and tactile feedbacks. Kinesthetic feedback provides force and torques that affect the motion and orientation of the user's hand or arm.  This type of feedback is able to convey multi-degree-of-freedom information to the user intuitively. However, systems that employ kinesthetic feedback are plagued by the problem of system stability. Issues such as time delay or inaccuracies in robot modeling might potentially destabilizes system that utilizes kinesthetic feedback. In addition, the mechanism required to provide kinesthetic force and torque feedback can often be large and mechanically complex.

To avoid some of the limitations of kinesthetic force and torque feedback, tactile feedback can be used. Tactile feedback stimulates cutaneous receptors in the user's skin. Tactile feedback can be used to create the perception of forces and torques without physically imparting these forces and torques to the user, and is therefore not destabilizing. Without the requirement to produce forces and torques, tactile feedback devices can also be designed to be compact, portable, and wearable. However, conveying multi-degree-of-freedom force and torque information via tactile feedback, in a manner that is intuitive to a human user, can be challenging.

This thesis focuses on the design, development, and experimental validation of a class of tactile feedback devices that provide feedback using local fingerpad skin deformation. Fingerpad skin deformation occurs in our daily interaction with objects, feels natural, and can intuitively express many degrees of freedom. Skin deformation tactile feedback can provide force and torque information through skin deformation cues on multiple fingers in a manner that is consistent with our interaction with external objects.


%%%%%%%%%%%%%%%%%%%%%%%%%%%%%%%%%%%%%%%%%%%%%%%%%%%%%%%%%%%%%%%%%%%%%%%%%%%%%%%%%%%%%%%%%%%%%%%%%%%%%%%%%%%%%%%%%%%%%%%%%%%%%%%%%%%%%%%%%%%%
%======================================================================
\section{Motivation}
%======================================================================
%%%%%%%%%%%%%%%%%%%%%%%%%%%%%%%%%%%%%%%%%%%%%%%%%%%%%%%%%%%%%%%%%%%%%%%%%%%%%%%%%%%%%%%%%%%%%%%%%%%%%%%%%%%%%%%%%%%%%%%%%%%%%%%%%%%%%%%%%%%%

% Haptic feedback is the provision of feedback related to the sense of touch. Traditionally, the focus of haptic feedback had been on providing kinesthetic force/torque feedback to the user. Kinesthetic force/torque feedback is intuitive and can be used to convey multiple degree-of-freedom information to the user. However, this type of feedback is sensitive to issues such as time delay or inaccuracies in robot modeling, which can potentially destabilize the system. Such destabilization may not be desirable in safety critical applications such as robot-assisted teleoperated surgery. Another method of providing haptic feedback is through tactile feedback. Tactile feedback provides information to the user by stimulating the cutaneous receptors in the users' skin. As tactile feedback does not generate forces to move users' hand/arm, such feedback lacks the physical constraint capability of force feedback, but does not face stability problems. Tactile feedback is also faced with other challenges such as the ability to convey dense information to the user in a manner that is intuitive and non-distracting to the user. The goal of this dissertation is to develop a new form of haptic feedback, based on tactile feedback, to convey forces/torque direction and magnitude information to the user in a way that can be intuitively understood by the user.

% This thesis focus on the development, implementation, and experimental validation of a tactile device that provides feedback based on local fingerpad skin deformation. Skin deformation is a type of tactile feedback received on the fingerpad when people uses a stylus-like tool to interact with surfaces in the environment. Friction between the fingerpad skin and the surface of the stylus-like tool generates tangential skin stretch, while pressure between the fingerpad skin and the stylus surface generates normal skin deformation.  As it occurs naturally in our daily interaction with objects, such a feedback can be easily understood by the user. The feedback can be extended to provide higher degree-of-freedom information by providing skin deformation cues to multiple fingerpads in a manner consistent with the experience in our daily interaction. Our results in this thesis showed that skin deformation tactile feedback can be used to provide effective sensory substitution of force information, or be used to provide additional force information to augment kinesthetic force feedback in both virtual and teleoperated manipulation task.

%======================================================================
%\subsection{Providing haptic feedback in surgical teleoperation systems}
%======================================================================

The primary motivation for this work is on providing haptic feedback for conveying force and torque information in applications where it is difficult to implement traditional kinesthetic force and torque feedback. An example of such application is robot-assisted minimally invasive surgery. In robot-assisted minimally invasive surgery, due to the strict safety requirement for surgical procedure, it is not desirable to incorporate kinesthetic force and torque feedback, which may cause instability in the system. In addition, even, with force feedback, due to the scaling down of motion between the master manipulator and the patient side manipulator, the display of forces on the master manipulator has to be scaled down in order to maintain the passivity and safety of the teleoperation system. Therefore, the feedback is severely degraded, which lowers the performance benefits that force feedback provides.

Other example application where kinesthetic force and torque feedback faces difficulty is in space teleoperation. In space teleoperation, due to the distance between the ground station (which houses the master manipulator) and the satellite (which houses the slave manipulator), there exists communication delays in the transmission of information between the two manipulators \cite{Sheridan1993}. This time delay will results in instability in the teleoperation system.

The above issues can be resolved through sensory substitution and/or augmentation of force and torque feedback. This can be done using visual, audio, or tactile feedback. While visual and audio feedback are viable candidates, these types of feedback modality may not be feasible in scenarios where vision and audio modality are already saturated with other information. Tactile sensory substitution and augmentation is promising as it uses the same sense of touch as kinesthetic force and torque feedback. In this dissertation, we proposed to use a form of tactile feedback called skin deformation feedback. We seek to design, implement, and experimentally validate skin deformation tactile haptic devices that provides force and torque information through skin deformation feedback.

% The above issues can be resolved using tactile feedback. We propose to provide sensory substitution or augmentation of force and torque feedback through tactile skin deformation. In this dissertation, we seek to design, implement, and experimentally validate haptic devices that provides haptic information through tactile skin deformation feedback.

%The primary motivation for this work is on providing haptic feedback in surgical teleoperation systems. Surgical procedures had evolved from open incision surgery, to minimally invasive surgery, and lately, to robotic minimally invasive surgery. In traditional open incision surgery, surgeons performed a large incision on the patient’s body and operate directly on the patient. Such an operation gives the best visual and haptic feedback, and is the most intuitive to the surgeon. However, the large incision in open incision surgery brings about many disadvantages to the patient, such as a longer hospital stay, and increased probability for infection. 

%Minimally invasive surgery (MIS) uses a long, slim tool which is inserted into patient’s body through a small incision. Using this approach, the intuitiveness of handling by the surgeon is traded for the safety benefits of the patient. Surgeons get reduced quality visual feedback, as manipulation is performed within the enclosed body of the patient, and they have to view the surgery site using 2D monitor display. Haptic feedback quality is also reduced as surgeons have to perform manipulation using long telescopic tools. In addition, due to the fulcrum effect, surgeons have to perform kinematic inversion (hand movement in the opposite direction to movement of the tool), which reduces the intuitiveness of the operation. However, the smaller incision for minimally invasive surgery means shorter hospital stay and reduced complication rate for the patient.

%Robotic minimally invasive surgery (RMIS) uses the same concept as minimally invasive surgery, but manipulation is done through teleoperation. Surgeons operate through a master console, which is used to control the patient side robot to perform minimally invasive surgical procedures. With this approach, surgeons get high quality visual feedback due to the use of 3D stereoscopic vision, and the ability to perform the task intuitively without any kinematic inversion. However, in today's surgical robotic system, haptic feedback is still lacking. By bring high quality haptic feedback to the surgeon, we can achieve manipulation intuitiveness similar to that of open incision surgery, but with increased patient safety brought about by minimally invasive surgery.

%The main reasons for the difficulty in providing haptic feedback in RMIS is the strict safety requirement in surgical procedure, and the scaling effect in RMIS. Bilateral force reflecting teleoperation closes the control loop between the master and the slave side manipulator and this inherently decreases the stability margin of the teleoperation system compared to unilateral teleoperation system. Bilateral force reflecting teleoperation is therefore sensitive to issues such as time delays or inaccuracies in robot modeling, which affects the stability and hence the overall safety of the system. The second reason is the scaling effect in RMIS. In RMIS, the movement of the surgeon's hands are scaled down on the patient side robot, allow surgeons to perform dedicate tasks. However, due to this scaling, the forces felt at the patient side robot has to be scaled down on the surgeon side in order to ensure passivity of the system. Therefore, even with force feedback, surgeons are only able to feel a soft version of the operating environment. 

%The above issues can be resolved through sensory substitution and augmentation of force feedback using tactile feedback. We propose to provide sensory substitution or augmentation of force and torque feedback through tactile skin deformation. In this dissertation, we seek to design, implement, and experimentally validate haptic devices that provides haptic information through tactile skin deformation feedback.

%%%%%%%%%%%%%%%%%%%%%%%%%%%%%%%%%%%%%%%%%%%%%%%%%%%%%%%%%%%%%%%%%%%%%%%%%%%%%%%%%%%%%%%%%%%%%%%%%%%%%%%%%%%%%%%%%%%%%%%%%%%%%%%%%%%%%%%%%%%%
%======================================================================
\section{Contributions}
%======================================================================
%%%%%%%%%%%%%%%%%%%%%%%%%%%%%%%%%%%%%%%%%%%%%%%%%%%%%%%%%%%%%%%%%%%%%%%%%%%%%%%%%%%%%%%%%%%%%%%%%%%%%%%%%%%%%%%%%%%%%%%%%%%%%%%%%%%%%%%%%%%%
We briefly summarize the major contributions of this dissertation as follows:
\begin{itemize}
\item We investigated the effect on the perception of stiffness of virtual surfaces when augmenting force feedback with skin stretch feedback, and proposed a model to explain this effect.
\item We developed a 3-Degree-of-Freedom (DoF) fingerpad skin deformation tactile device. Through human user studies, we demonstrated the feasibility of using this device for sensory substitution and augmentation of forces in 3-DoF.
\item We developed a 6-Degree-of-Freedom (DoF) fingerpad skin deformation tactile device. Through human user studies, we demonstrated the feasibility of using this device for sensory substitution and augmentation of force and torques in 6-DoF.
\item We developed a novel control algorithm for the rendering of skin deformation feedback based on force and torque information.
\item We integrated the skin deformation tactile devices with a surgical robotic system with skin deformation tactile feedback. We evaluate the combined system through human user studies, in which participants uses the combined system to perform simulated surgical tasks.
\end{itemize}

%%%%%%%%%%%%%%%%%%%%%%%%%%%%%%%%%%%%%%%%%%%%%%%%%%%%%%%%%%%%%%%%%%%%%%%%%%%%%%%%%%%%%%%%%%%%%%%%%%%%%%%%%%%%%%%%%%%%%%%%%%%%%%%%%%%%%%%%%%%%
%======================================================================
\section{Prior Work}
%======================================================================
%%%%%%%%%%%%%%%%%%%%%%%%%%%%%%%%%%%%%%%%%%%%%%%%%%%%%%%%%%%%%%%%%%%%%%%%%%%%%%%%%%%%%%%%%%%%%%%%%%%%%%%%%%%%%%%%%%%%%%%%%%%%%%%%%%%%%%%%%%%%

%======================================================================
\subsection{Tactile Sensory Substitution}
%======================================================================
Sensory substitution is the transformation of the characteristics of one sensory modality into another sensory modality. In force-feedback sensory substitution, the force is replaced by other sensory modalities that are used to convey force magnitude and/or direction information to the user. Tactile feedback can be used for force-feedback sensory substitution, with the main forms of tactile feedback being vibrotactile, skin stretch, and normal skin deformation. Vibrotactile feedback has been used for sensory substitution in a wide variety of applications, e.g. providing grip force information for prosthetic applications \cite{Walker2014}, interaction force information in teleoperated assembly \cite{Debus2001}, and tissue interaction force information in robot-assisted surgery \cite{Schoonmaker2006}. The main drawback of vibrotactile feedback is the difficulty of conveying both force direction and magnitude information together. Tappeiner et al. \cite{Tappeiner2009} showed that directional cues can be conveyed using asymmetric vibration. However, their work is currently limited to in-plane direction rendering, and it is not known whether similar concepts can be used to convey 3-DoF directional cues to the user. With traditional vibrating actuator such as eccentric rotating mass motor or linear resonant actuator, multiple actuators can be placed side-by-side to convey direction information through sensory saltation \cite{Weber2011}. Using this method, however, the actuators have to be spaced some distance apart to allow for participants to discriminate between the different vibrating actuators. Such temporal-based direction display is also not practical for use scenarios where the interaction force direction can change rapidly. Another drawback of vibrotactile feedback is that the sensitivity of the skin to ongoing vibration stimuli decreases over time \cite{Bensma2005}. Vibrotactile feedback can also be distracting and uncomfortable over long periods of usage \cite{Okamoto2011}.

Compared to vibrotactile feedback, skin stretch tactile feedback has the advantage of being able to convey both magnitude and directional information at the same time. Gleeson et al. \cite{Gleeson2010} and Guinan et al. \cite{Guinan2012} used servo motors to move high friction surfaces across users' fingerpads to convey translation and rotation navigation information. They have also designed a device that stretches the skin of the palm of the user to convey rotational inertia \cite{Guinan2014}. These devices are shown in Fig.~\ref{fig:Chap_1_SkinStretchDevices}(a) and (b). In addition to translational skin stretch, Bark et al. \cite{Bark2010} designed a rotational skin stretch device, shown in Fig.~\ref{fig:Chap_1_SkinStretchDevices}(c), to convey proprioceptive information to users for gait rehabilitation. 
%Schorr et al. \cite{Schorr2013} used a similar skin stretch device to show that users are able to interpret skin stretch rate to discriminate virtual surfaces with different stiffness values.

\begin{figure*}[h]
\centering
\includegraphics[scale=0.53]{./Figures/Chapter_1/Chap_1_SkinStretchDevices.pdf}
\caption{\small Skin stretch feedback devices: (a) Fingerpad skin stretch device (b) Device that stretches the palm of user's hand (c) Rotational skin stretch device. Photos adapted from \cite{Guinan2012}, \cite{Guinan2014}, and \cite{Bark2010} respectively.}
\label{fig:Chap_1_SkinStretchDevices}
\end{figure*}

Force information can also be conveyed to the user by application of cutaneous normal force to the users' fingerpads. Minamizawa et al. \cite{Minamizawa2010} developed a device that used dual motors to apply normal forces to the users' fingerpad. Prattichizzo et al. \cite{Prattichizzo2012} extended the idea with a device that applies 3-DoF cutaneous forces to the fingerpad. Using a feedback method termed as ``sensory subtraction", kinesthetic forces are subtracted from the combination of kinesthetic and cutaneous sensations present during normal interaction, leaving only the cutaneous sensations that are fed back to the user using a tactile feedback device. They used the device to perform a needle insertion task, and showed that superior performance is achieved using cutaneous normal force feedback compared to visual feedback. With the same device, Meli et al. \cite{Meli2014} performed a task in which users insert rings onto pegs, emulating the Peg-board module of the da Vinci Surgical Simulator (Intuitive Surgical, Inc.). They found that sensory subtraction achieved performance better than the traditional visual and audio sensory substitution methods. 

\begin{figure*}[h]
\centering
\includegraphics[scale=0.53]{./Figures/Chapter_1/Chap_1_CutaneousNormal.pdf}
\caption{\small Wearable cutaneous normal force feedback device: (a) 1-DoF device (b) 3-DoF device .}
\label{fig:Chap_1_CutaneousNormal}
\end{figure*}

%======================================================================
\subsection{Tactile Sensory Augmentation}
%======================================================================
The same tactile modality used for sensory substitution can be used for sensory augmentation. Researchers have looked at the effect of such tactile augmentation of force feedback on user perception. Okamoto et al. \cite{Okamoto2011} used vibration on the fingerpad to bias the perceived viscous and inertia properties of an object. In addition, Okamura et al. \cite{Okamura2001} and Kuchenbecker et al. \cite{Kuchenbecker2006} rendered vibration in conjunction with forces using a force-feedback device to increase the perception of hardness of an object. Besides vibration, skin stretch feedback has also been used in conjunction with force feedback to bias the perception of friction \cite{Provancher2009} of a haptically rendered virtual surface.

Several works have also looked into the effect of sensory augmentation on task performance. In many cases, adding tactile feedback to force feedback achieved performance better than the case when either tactile or force feedback is used alone. Augmenting force feedback with vibration feedback decreases the contact force error in a path-tracing task \cite{Debus2001_2} and reduces reaction time to tissue puncture in a teleoperated needle insertion task \cite{Kontarinis1996}, while augmenting force feedback with skin deformation feedback (normal skin deformation or tangential skin stretch) decreases the penetration into a forbidden region in a needle insertion task \cite{Pacchierotti2014}\cite{Tirmizi2013} and improves accuracy in a direction identification task \cite{Gwilliam2013}.

In this dissertation, we combined the idea of providing tangential skin stretch and normal forces to the user's fingerpad to develop tactile devices that can provide skin deformation cues to the fingerpad. This skin deformation cues are used to provide force and torque information to substitute or augment kinesthetic force and torque feedback.

%======================================================================
\subsection{Other Forms of Tactile Feedback}
%======================================================================
Other forms of tactile feedback had been looked into, not for tactile sensory substitution or augmentation of force/torque feedback, but for research into human perception, or to provide tactile feedback to systems where such type of feedback are absent. Hayward and Cruz-Hernandez  \cite{Hayward2000} constructed an array of piezoelectric actuators closely packed within a membrane to create lateral skin stretch. Their device generates a programmable stress field within the fingerpad. Drewing et al. \cite{Drewing2005} also constructed a multiple-contact shear display using mechanical linkages and servo motors, as shown in Fig.~\ref{fig:Chap_1_OtherTactile}(a). Using this device, they performed perceptual experiments to determine the human sensitivity for tactile movement perception, and found that the direction perception Just Noticeable Difference (JND) is no better than 14 degrees for all subjects.

Winfield et al. \cite{Winfield2007} designed a device, called the T-pad, which can rapidly alter the friction property of a surface through out-of-plane vibration. By generating in-plane vibration, and by rapidly changing the property of the surface between low and high friction through out-of-plane vibration, the device can modulate the amount of shear force applied to the subject. Such device can be used to render tactile feedback in devices such as tablets and smartphones. Chubb et al. \cite{Chubb2010} improved on the design of the original T-pad to increase the in-plane vibration frequency, and they found that this helped to improve the 3D-edge rendering capability compared to the original T-pad. 

\begin{figure*}[h]
\centering
\includegraphics[scale=0.50]{./Figures/Chapter_1/Chap_1_OtherTactile.pdf}
\caption{\small(a) Multiple-contact shear display using mechanical linkages and servo motos (b) Contact location display (c) 2-D slip display. Photos adapted from \cite{Drewing2005}, \cite{Park2012}, and \cite{Webster2005} respectively.}
\label{fig:Chap_1_OtherTactile}
\end{figure*}

Provancher et al. \cite{Provancher2005} designed a tactile device, called the contact location display (Fig.~\ref{fig:Chap_1_OtherTactile}(b)), which uses tactile feedback to render the location of the contact centroid moving on the user's fingertip. They found, through human perception experiment with virtual environment, which users are able to distinguish between objects of different curvature, as well as the interaction types using such contact location display.

Webster et al. \cite{Webster2005} uses two DC motors to drive a ball using contact friction, as shown in Fig.~\ref{fig:Chap_1_OtherTactile}(c). The ball is positioned under the user's fingerpad, and rotation of the ball reproduces the sensations of sliding contact and slip. They found, through human subject study, that participants are able to use the combination of slip and force feedback to complete a virtual paper manipulation task with lowered applied force compared to force feedback alone.

%======================================================================
\subsection{Feedback of force information in Surgical Teleoperation Systems}
%======================================================================

\subsubsection{Force and Tactile feedback in robotic surgery}
Kinesthetic force feedback had been found to be useful in surgery. The force feedback helps to reduce interaction forces \cite{Wagner2007} while improving task performance such as tissue characterization \cite{Tholey2005} and suture manipulation \cite{Kitagawa2005}\cite{Talasaz2012}. However, due to several issues, which includes the stability of the system, current surgical teleoperation systems do not incorporate force feedback \cite{Okamura2009}. To overcome this issue, McMahan et al. \cite{McMahan2011} had looked into the feedback of interaction information through vibrotactile feedback in surgical teleoperation systems. Instead of relaying the whole spectrum of interaction force information, they focus particularly on the sensing and feedback of high-frequency vibration during the performance of a task. Accelerometers were placed on the patient-side robot of a clinical da Vinci system to sense the tool acceleration, and this information is fed back and displayed through vibrotactile actuators on the master tool manipulator. They found that such tactile feedback improves the surgeon's concentration and situation awareness, while maintaining task performance when working on a teleoperated surgical task.

Besides conveying information related to interaction with objects in the surgical environment, King et al. \cite{King2009} conveyed tactile information related to grasping. They used a piezoresistive force sensor to measure the grip force on the patient-side robot of the da Vinci surgical teleoperation system, and pneumatic balloon actuators on the master side manipulator to feed back the grip force information to the user. The display of gripping force information through such tactile approach decreases the gripping force during execution of a peg-transfer task.

Prattichizzo et al. \cite{Prattichizzo2012} proposed another approach to providing tactile feedback to the user, in which they remove the kinesthetic force component and provide only the cutaneous tactile sensation to the user. They does this through a wearable device that is worn around the users' fingerpad. The wearable device consists of small dc motor actuators that are able to create cutaneous forces in all direction on your fingerpad. They validated this approach through a virtual peg-transfer task \cite{Meli2014}, and found that such an approach helped to reduce insertion forces and time when compared to the case with just visual feedback.

Other methods had been developed for conveying tactile information in tasks that are surgically motivated, such as palpation and lump detection, though these systems had not been integrated and tested with actual surgical teleoperation systems. Gwilliam et al. \cite{Gwilliam2013} developed an air jet lump display that uses an air jet directed through an aperture to create sensations of lumps of different hardness and size. Serio et al. \cite{Serio2013} had developed a concentric tube tactile display that is able to vary the contact area spread rate during palpation to display surfaces of different compliance. Such methods had also been used by Yazdian et al. \cite{Yazdian2014} who uses a tilting plate, and Kimura et al. \cite{Kimura2010} who uses a flexible surface to wrap around the fingers to vary the contact area spread rate during palpation.

\subsubsection{Visual feedback in robotic surgery}
Besides conveying force information through the haptic modality, Gwilliam et al. \cite{Gwilliam2009} and Reiley et al. \cite{Reiley2007} implemented a system that is able to graphically display interaction force information to the user, shown alongside the surgeons' view of the surgical environment through either a force bar display or a visual color indicator. Such vision-based display of interaction force information decreased interaction forces and lowered surgeon's mental workload during performance of surgical task. Tavakoli et al. \cite{Tavakoli2005} performed telemanipulated suturing experiment with visual force feedback and found the same decrease in interaction force. However, he found that such performance benefits are only obtained if the surgeons paid attention to the visual force feedback, resulting in higher mental workload for the surgeons.

%%%%%%%%%%%%%%%%%%%%%%%%%%%%%%%%%%%%%%%%%%%%%%%%%%%%%%%%%%%%%%%%%%%%%%%%%%%%%%%%%%%%%%%%%%%%%%%%%%%%%%%%%%%%%%%%%%%%%%%%%%%%%%%%%%%%%%%%%%%%
%======================================================================
\section{Dissertation Overview}
%======================================================================
%%%%%%%%%%%%%%%%%%%%%%%%%%%%%%%%%%%%%%%%%%%%%%%%%%%%%%%%%%%%%%%%%%%%%%%%%%%%%%%%%%%%%%%%%%%%%%%%%%%%%%%%%%%%%%%%%%%%%%%%%%%%%%%%%%%%%%%%%%%%
This thesis consists of six chapters. In chapter 1, which is this introduction, we presented the motivation for our research of using tactile feedback for sensory substitution and augmentation of force feedback. We also presented prior work on the various types of tactile feedback devices and prior work on providing interaction force information in robot-assisted minimally invasive surgery.

Chapter 2 describes how tangential skin stretch feedback can rendered together with force feedback to increase stiffness perception. We performed human user experiment to determine how human perception of stiffness of objects are affected by additional skin stretch cues. We also derived a model utilizing the framework of multi-sensory integration to characterize the shift in stiffness perception due to skin stretch augmentation.

Our results from Chapter 2 highlighted the promising approach of using fingerpad skin deformation for force feedback substitution and augmentation. In Chapter 3, we developed a  3-Degree-of-Freedom (DoF) skin deformation tactile device that is able to communicate 3-DoF force information for sensory substitution or augmentation of force feedback. We performed human user experiment that evaluated participants' ability to interpret the 3-DoF force information to locate a feature in a virtual environment, and how participants' task performance can be improved by augmenting force feedback with skin deformation feedback and vice versa.

While skin deformation tactile feedback can be used to convey 3-DoF force information, similar approach can be used to convey 3-DoF torque information, or 6-DoF force and torque information to the user. These information can be used for sensory substitution and augmentation of forces and torques. In Chapter 4, we designed and built a skin deformation tactile device that can convey 6-DoF force and torque information. We performed a human user experiment to verify the device's capability as well as participants' ability to interpret the 6-DoF force and torque information to perform a peg-in-hole insertion tasks. 

In Chapter 5, we integrated the 3-DoF skin deformation tactile device with a surgical teleoperation system. The viability of using skin deformation tactile feedback, as well as its usefulness, is evaluated via a human user experiment in which participants performed surgically related tasks with force and/or skin deformation tactile feedback.

Chapter 6 summarizes the results of this research, reviews the contribution made in this dissertation, and provides suggestions for future work.