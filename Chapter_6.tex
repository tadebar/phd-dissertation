%!TEX root = PhD_Thesis.tex
\chapter[Conclusions and Future Work]{Conclusions and Future Work}
This dissertation has presented the design, implementation and experimental validation of methods for ultrasound-image-guided needle steering in percutaneous ablation of liver tumors, including (1) a method for segmenting steerable needles from 3D Doppler ultrasound data, (2) an articulated-tip steerable needle design that achieves clinically useful curvature in liver tissue, and (3) an estimation scheme and human interface design that allows manual teleoperation of a steerable needle using freehand 3D ultrasound. 

% Segmentation
The described segmentation method uses ultrasound Doppler imaging, in combination with high-frequency low-amplitude vibration of the steerable needle, to greatly simplify a challenging segmentation problem. Using the Doppler method, unsophisticated image processing methods can localize the needle with run times of approximately 60 ms per image, and error on the order of 1~mm to 2~mm for relevant needle configurations. This segmentation algorithm has been shown to be robust to curvature, image orientation, and vibration frequency. Experiments in \textit{ex vivo} liver tissue have demonstrated that this method is accurate to within 1.2~mm of manual segmentation on average. Combining this segmentation method with a replanning-type control algorithm allows a needle steering robot to reach a simulated target with an average error below 2~mm. 

There are several potential directions for improvement of the Doppler segmentation method. Information on mechanical properties of the needle, such as minimum bending radius, could be used to improve the initial 2D segmentation, for example by limiting the search region for the needle cross section based on the maximum possible needle curvature. Combining data from multiple ultrasound scan modes (\textit{e.g.}, Doppler data, B-mode data, RF scanline data, and elastography data) could also potentially improve the accuracy and consistency of segmentation. For example, the Doppler data could be interpreted as a highlighter, allowing a more costly analysis of the B-mode data inside a region of interest. A preliminary implementation of this concept showed promising results, with millisecond runtimes and sub-millimeter accuracy~\cite{Greer2014}. The characteristics of the Doppler response might be improved by vibrating the needle tip itself, rather than the proximal shaft of the needle. One approach would be to place a small magnet in the shaft of the needle, and apply an alternating magnetic field using an electromagnet outside the body. A preliminary implementation of this concept increased the Doppler response at the tip of the needle compared to base vibration~\cite{Cabreros2014}.

%The rest...


In this paper, we examined the ability of bent-tip steerable needles to follow tightly curved paths in liver tissue. A simplified FE model of bent-tip needle steering suggested that selection of bent-tip geometry could significantly improve needle curvature. This was confirmed by experiments in \textit{ex vivo} porcine liver tissue. Specifically, with tip length $l =$~12~mm and tip angle $\alpha =$~45~degrees, we found average radius of curvature to be below 50~mm in liver tissue, which is a significant improvement for a 0.8-mm tubular Nitinol needle compared to values reported in the literature. We also described the design of an articulated-tip steerable needle, which uses a miniaturized cable-driven rotary joint to articulate the needle tip between straight and bent configurations. This design allows the needle to have a more asymmetric tip, which results in better curvature, while the straight configuration also allows the needle to pass through an introducer sheath and rotate without extensive tissue deformation. Experimental validation testing demonstrates that the articulated-tip needle is able to achieve tighter curvature in liver tissue than comparable existing bent-tip needles. In future work, we will combine the articulated-tip needle design with our previously described methods for image-guided control~\cite{Adebar2014}. We will also integrate a functional ablation element into the articulated-tip design. The combination of highly steerable needles, closed-loop image-guided control, and a functional ablation element will allow us to test image-guided needle steering for RFA of liver tumors in a live porcine model.

An interesting issue for future consideration is whether existing kinematic models of needle steering, such as the unicycle and bicycle models~\cite{Park2005,Webster2006}, accurately capture the behavior of these tightly curving steerable needles in tissue. With tighter curvature there is more relaxation of the surrounding tissues and more deviation from the ideal piece-wise circular paths, as seen in Fig.~\ref{fig:InsertAndTurn}.

We have demonstrated closed-loop robotic needle steering in biological tissue using medical image feedback with a clinically realistic level of measurement noise. A recursive estimation approach allows a robotic system to steer a needle along a 3D path, and position the needle tip at a target with average error of about 2~mm. We have also described experimental measurement of the variability of asymmetric-tip needle steering in \textit{ex vivo} biological tissue. One limitation of the current approach is the reliance on kinematic models for steerable needle motion which have mostly been validated in homogeneous artificial tissues. In our future work, we will evaluate whether such kinematic models accurately capture the behavior of asymmetric-tip steerable needles in biological tissue. Overall, this estimation scheme, in combination with improved methods for needle segmentation, is an important step towards future \textit{in vivo} needle steering.

Design of a practical needle steering robot that is appropriate for \textit{in vivo} testing is another important direction for future research directed towards clinical needle steering. Since the vast majority of prior needle steering methods have been evaluated in artificial or \textit{ex vivo} tissues, needle steering robots described to date have not generally been appropriate for actual clinical environments. We will thus modify our robot design to delineate disposable components, sterilizable components, and components that must be draped or otherwise isolated.