%!TEX root = PhD_Thesis.tex
\chapter[Conclusions and Future Work]{Conclusions and Future Work}
This dissertation has presented the design, implementation and experimental validation of methods for ultrasound-image-guided needle steering in percutaneous ablation of liver tumors, including (1) a method for segmenting steerable needles from 3D Doppler ultrasound data, (2) an articulated-tip steerable needle design that achieves clinically useful curvature in liver tissue, (3) an estimation scheme that allows infrequent noisy ultrasound measurements to be used for closed-loop control, and (4) an interface and system design that allows human-in-the-loop control of a steerable needle using freehand 3D ultrasound. 

In the remainder of this chapter, we briefly review the results and contributions in each of these areas, and describe ideas for future work. Finally, we conclude with possible extensions of the work to other areas of medical robotics.

\section{Doppler Segmentation of Steerable Needles}
In Chapter 2, we described using ultrasound Doppler imaging, in combination with high-frequency low-amplitude vibration of a steerable needle, to greatly simplify a challenging segmentation problem. Using the Doppler method, unsophisticated image processing methods can localize the needle with runtimes of milliseconds per image, and error on the order of 1~mm to 2~mm for relevant needle configurations. This segmentation algorithm has been shown to be robust to curvature, image orientation, and vibration frequency. Experiments in \textit{ex vivo} liver tissue have demonstrated that this method is accurate to within 1.2~mm of manual segmentation on average. Combining this segmentation method with a simple replanning-type control algorithm allows a needle steering robot to reach a simulated target with an average error below 2~mm. 

There are several potential directions for improvement of the Doppler segmentation method. Knowledge of mechanical properties of the needle, such as minimum bending radius, could be used to improve the initial 2D segmentation, for example by limiting the search region for the needle cross section based on the maximum possible needle curvature. Combining data from multiple ultrasound scan modes (e.g., Doppler data, B-mode data, RF scanline data, and elastography data) could also potentially improve the accuracy and consistency of segmentation. For example, the Doppler data could be interpreted as a highlighter, allowing a more costly analysis of the B-mode data inside a region of interest. A preliminary implementation of this concept showed promising results, with millisecond runtimes and sub-millimeter accuracy~\cite{Greer2014}. The characteristics of the Doppler response might be improved by vibrating the needle tip itself, rather than the proximal shaft of the needle. One approach would be to place a small magnet in the shaft of the needle, and apply an alternating magnetic field using an electromagnet outside the body. A preliminary implementation of this concept increased the Doppler response at the tip of the needle compared to base vibration~\cite{Cabreros2014}. It might also be possible to induce a standing wave in the needle shaft, which could improve visibility in the Doppler data and provide additional information on needle insertion depth.

\section{Needle Design and Curvature}
In Chapter 3, we examined the ability of bent-tip steerable needles to follow clinically relevant curved paths in liver tissue. An analysis of contrast-enhanced CT data was completed to define a procedure-specific requirement for needle curvature in ablation of liver tumors. This study showed that a radius of curvature of 50~mm or less is necessary to reach the majority of the liver volume without injuring the liver capsule or major vasculature. A simplified FE model of bent-tip needle steering suggested that selection of bent-tip geometry could improve needle curvature to acceptable levels. This was confirmed by experiments in \textit{ex vivo} porcine liver tissue. Specifically, with tip length $l =$~12~mm and tip angle $\alpha =$~45~degrees, we found average radius of curvature to be below 50~mm in liver tissue, which is a significant improvement for a 0.8-mm tubular Nitinol needle compared to values reported in the literature. We also described the design of an articulated-tip steerable needle, which uses a miniaturized cable-driven rotary joint to articulate the needle tip between straight and bent configurations. This design allows the needle to have a more asymmetric tip, which results in tighter curvature, while the straight configuration also allows the needle to pass through an introducer sheath and rotate without extensive tissue deformation. Experimental validation testing demonstrated that the articulated-tip needle is able to achieve tighter curvature in liver tissue than comparable existing bent-tip needles. 

Although the articulated-tip design showed promising results in biological tissue, two main problems with needle design remain to be solved. The first problem is the robustness of the articulated design, which will need to be improved before such needles can be applied in a cadaver or animal model. As discussed in Chapter~3, the current plastic hinges fail too frequently to be applied in simulated clinical testing. Micromachining or additive manufacturing methods should be able to produce more robust metal hinges. Indeed, we explored such methods, but were unable to produce an acceptable prototype before moving on to the system testing described in Chapter~5. This led to the use of Nitinol flexure tips with exaggerated geometry. The second problem is the integration of a functional ablation element into the steerable needle.

There are multiple concepts for incorporating an actual ablation element into a steerable needle. The simplest solution would be to use the steerable needle itself as an electrode. Most of the needle shaft would be insulated, while a conducting tip would be used to deposit radiofrequency energy into the target tissue. This would create a very localized ablation, and multiple passes would be required for large tumors. Alternatively existing RFA probes, with their multiple expanding electrode tines, could be used. These probes are generally much thicker and stiffer than steerable needles, but the steerable needle could be used as a type of guidewire in an initial insertion. The RFA probe could then be introduced coaxially over the steerable needle. Although this would cause deformation of the surrounding tissue, the goals of avoiding obstacles and reaching obscured targets would still be achieved. Applying tubular steerable needles and alternative ablation technologies might provide the best solution. The tubular steerable needles described in Chapter 3 and Chapter 5 could be used directly for ethanol ablation of confined tumors, since this requires only a very small lumen for fluid delivery. Recent advances in microwave ablation antenna design~\cite{Luyen2014,Luyen2015} may also make it possible to integrate a functional microwave ablation element directly into existing tubular needles. 

An interesting issue for future consideration is whether existing kinematic models of needle steering, such as the unicycle and bicycle models~\cite{Webster2006,Park2005}, accurately capture the behavior of tightly curving steerable needles in tissue, as described in Chapter~3. With tighter curvature there is more relaxation of the surrounding tissues and more deviation from the ideal piece-wise circular paths, as seen in Fig.~\ref{fig:InsertAndTurn}.

\section{UKF Estimation Scheme}
In Chapter 4, we described a recursive estimation approach based on an unscented Kalman filter. Experimental measurement of the variability of asymmetric-tip needle steering in \textit{ex vivo} biological tissue was completed in order to formulate the UKF. Results in simulations and bench-top testing showed this filter allows a needle steering robot to autonomously position the needle tip in a small workspace, with average error of about 2~mm.

In our current implementation, the UKF only receives feedback after a complement measurement of the steerable needle's tip pose, either through an automatic segmentation or manual localization. This currently requires a complete ultrasound sweep of the needle, with many images processed to yield only one measurement. Interestingly, all images of the needle shaft do provide some information on the tip's pose. In future work, it might be possible to construct a single probabilistic framework which would incorporate each ultrasound frame containing the needle shaft as a measurement in order to refine the tip pose estimate.

\section{Human-in-the-Loop Control} 
In Chapter 5, we described human-in-the-loop control of a needle steering robot, using freehand 3D ultrasound imaging to define targets and provide control feedback. The system described in this chapter combined our methods for imaging, needle design, and estimation, in what we believe to be a clinically realistic implementation. Comparison with an electromagnetic tracking system showed our UKF estimation scheme is able to track the position of the steerable needle tip with an average error of approximately 4~mm based on intermittent manual localizations. Validation testing in a clinical scenario using a porcine cadaver showed that our system is able to steer to a simulated target with an error of approximately 4~mm for the best-case target. Tip placement error increased dramatically for more difficult targets, and appeared to be a result of poor curvature performance where the needle's radius of curvature was higher than expected over portions of the trial. 

Our current implementation of human-in-the-loop control does not provide any haptic feedback to the user, unlike the previous schemes described by Majewicz and Okamura~\cite{Majewicz2013}, and Romano et al.~\cite{Romano2007}. Incorporating a force sensor at the proximal end of the needle, and providing haptic, auditory, or visual feedback of insertion force, might allow earlier detection of the tip catching and needle buckling seen in our testing. 

Future work should also examine the mechanical behavior of bent-tip steerable needles in \textit{in situ} cadaveric porcine liver specimens. The results of Chapter 3 demonstrated that bent-tip steerable needles can achieve clinically relevant radius of curvature in liver tissue in a bench-top setting. However, the results of the pre-clinical testing in Chapter 5 showed that inconsistent steering behavior in more realistic tests can lead to unacceptably large steering error. Practical considerations such as sharper needle tips or a better method for connecting with the introducer needle might improve steering behavior. Alternatively, a different needle steering implementation which relies less on the mechanical interaction of the needle tip and the surrounding tissues (e.g., the active cannula robots described by Sears and Dupont~\cite{Sears2006}) might achieve better results in this application. The difference in steering performance between our bench-top and cadaver experiments reinforces the importance of validating needle steering techniques in realistic biological tissues and clinical environments. 

\section{Extensions}
Although this dissertation has focused narrowly on steerable needles and percutaneous ablation of liver tumors, several of the described methods are relevant to other areas of medical robotics. The Doppler segmentation technique described in Chapter 2 could also be applied to robotic catheters, concentric-tube robots, or robotic manipulators in other ultrasound-guided interventions. The simplified FE modeling approach described in Chapter 3 might be a useful starting point for future studies examining other robotic manipulations of tissue, such as grasping or retracting in minimally invasive surgery. With some modifications, the workspace analysis technique described in Chapter 3 could be applied to other interventional devices and other organ systems. For example, investigators applying continuum robots to steer through vasculature or lung might examine the reachable organ volume as function of the maximum reflexive angle of the manipulator. Finally, the UKF estimation scheme could serve as a framework for image guidance in a variety of robotic interventions. Medical imaging produces measurements at lower frequency and with greater noise than typical robotic sensors. Incorporating a system model into such an estimation scheme allows this medical imaging to still produce a useful result.




